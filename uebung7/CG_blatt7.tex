\documentclass[12pt,pdftex,a4paper]{article}
\usepackage[ngerman]{babel}
\usepackage[utf8]{inputenc}
\usepackage{amsmath}
\usepackage{amssymb}
\usepackage{ulem}
%\usepackage{bbm}
\usepackage{array}
\usepackage{marvosym}
\usepackage{color}
\usepackage{hhline}
\newcommand{\bbN}{\mathbbm{N}}
\newcommand{\bbR}{\mathbbm{R}}
\newcommand{\bbZ}{\mathbbm{Z}}
\newcommand{\bbI}{\mathbbm{I}}
\newcolumntype{L}[1]{>{\raggedright\let\newline\\\arraybackslash\hspace{0pt}}m{#1}}
\newcolumntype{C}[1]{>{\centering\let\newline\\\arraybackslash\hspace{0pt}}m{#1}}
\newcolumntype{R}[1]{>{\raggedleft\let\newline\\\arraybackslash\hspace{0pt}}m{#1}}
\usepackage[pdftex]{graphicx}
\usepackage{listings}
\lstset{language=Python,basicstyle=\footnotesize}
\usepackage{pdfpages}
\usepackage{booktabs}
\PassOptionsToPackage{hyphens}{url}
\usepackage{hyperref}

\begin{document}
\title{ Grundlagen der Computergrafik,\\ Blatt 7}
\author{Lukas Baur, 3131138\\
		Felix Bühler, 2973410\\
		Marco Hildenbrand, 3137242}
\maketitle
\section*{Aufgabe 1}
\subsection*{1) Translation und Skalierung}
\[
M_1=
\begin{bmatrix}
2 & 0 & 0\\
0 & 1 & 0\\
0 & 0 & 1
\end{bmatrix}
*
\begin{bmatrix}
1 & 0 & 3\\
0 & 1 & 4\\
0 & 0 & 1
\end{bmatrix}
=
\begin{bmatrix}
2 & 0 & 3\\
0 & 1 & 4\\
0 & 0 & 1
\end{bmatrix}
\]

\subsection*{2) Translation und Rotation}
\[
M_2=
\begin{bmatrix}
cos(-30) & -sin(-30) & 0\\
sin(-30) & cos(-30) & 0\\
0 & 0 & 1
\end{bmatrix}
*
\begin{bmatrix}
1 & 0 & 4\\
0 & 1 & 0\\
0 & 0 & 1
\end{bmatrix}
=
\begin{bmatrix}
\frac{\sqrt{2}}{3} & \frac{1}{2} & 2\sqrt{3}\\
-\frac{1}{2} & \frac{\sqrt{2}}{3} & -2\\
0 & 0 & 1
\end{bmatrix}
\]

\subsection*{3) Scherung}
\[
M_3=
\begin{bmatrix}
1 & 1 & 0\\
0 & 1 & 0\\
0 & 0 & 1
\end{bmatrix}
\]

\subsection*{4) Rotation}
\[
M_4=
\begin{bmatrix}
1 & 0 & 2\\
0 & 1 & 2\\
0 & 0 & 1
\end{bmatrix}
*
\begin{bmatrix}
cos(-60) & -sin(-60) & 0\\
sin(-60) & cos(-60) & 0\\
0 & 0 & 1
\end{bmatrix}
*
\begin{bmatrix}
1 & 0 & -2\\
0 & 1 & -2\\
0 & 0 & 1
\end{bmatrix}
=
\begin{bmatrix}
\frac{1}{2} & \frac{\sqrt{3}}{2} & 1-\sqrt{3}\\
-\frac{\sqrt{3}}{2} & \frac{1}{2} & 1+\sqrt{3}\\
0 & 0 & 1

\end{bmatrix}
\]

\section*{Problem 2}

\end{document}


