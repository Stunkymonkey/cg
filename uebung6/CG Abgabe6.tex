\documentclass{article}

\usepackage{fancyhdr} % Required for custom headers
\usepackage{lastpage} % Required to determine the last page for the footer
\usepackage{extramarks} % Required for headers and footers
\usepackage{graphicx} % Required to insert images
\usepackage{lipsum} % Used for inserting dummy 'Lorem ipsum' text into the template

\usepackage[ngerman]{babel}
\usepackage[utf8]{inputenc}

% Margins
\topmargin=-0.45in
\evensidemargin=0in
\oddsidemargin=0in
\textwidth=6.5in
\textheight=9.0in
\headsep=0.55in 

\linespread{1.1} % Line spacing

% Set up the header and footer
\pagestyle{fancy}
\lhead{\hmwkAuthorName} % Top left header
\rhead{\hmwkClass\ (\hmwkClassInstructor): \hmwkTitle} % Top right header
\chead{} % Top right header
\lfoot{\lastxmark} % Bottom left footer
\cfoot{} % Bottom center footer
\rfoot{Page\ \thepage\ of\ \pageref{LastPage}} % Bottom right footer
\renewcommand\headrulewidth{0.4pt} % Size of the header rule
\renewcommand\footrulewidth{0.4pt} % Size of the footer rule

\setlength\parindent{0pt} % Removes all indentation from paragraphs

%----------------------------------------------------------------------------------------
%	DOCUMENT STRUCTURE COMMANDS
%	Skip this unless you know what you're doing
%----------------------------------------------------------------------------------------

% Header and footer for when a page split occurs within a problem environment
\newcommand{\enterProblemHeader}[1]{
\nobreak\extramarks{#1}{#1 continued on next page\ldots}\nobreak
\nobreak\extramarks{#1 (continued)}{#1 continued on next page\ldots}\nobreak
}

% Header and footer for when a page split occurs between problem environments
\newcommand{\exitProblemHeader}[1]{
\nobreak\extramarks{#1 (continued)}{#1 continued on next page\ldots}\nobreak
\nobreak\extramarks{#1}{}\nobreak
}

\setcounter{secnumdepth}{0} % Removes default section numbers
\newcounter{homeworkProblemCounter} % Creates a counter to keep track of the number of problems

\newcommand{\homeworkProblemName}{}
\newenvironment{homeworkProblem}[1][Aufgabe \arabic{homeworkProblemCounter}]{ % Makes a new environment called homeworkProblem which takes 1 argument (custom name) but the default is "Problem #"
\stepcounter{homeworkProblemCounter} % Increase counter for number of problems
\renewcommand{\homeworkProblemName}{#1} % Assign \homeworkProblemName the name of the problem
\section{\homeworkProblemName} % Make a section in the document with the custom problem count
\enterProblemHeader{\homeworkProblemName} % Header and footer within the environment
}{
\exitProblemHeader{\homeworkProblemName} % Header and footer after the environment
}

\newcommand{\problemAnswer}[1]{ % Defines the problem answer command with the content as the only argument
\noindent\framebox[\columnwidth][c]{\begin{minipage}{0.98\columnwidth}#1\end{minipage}} % Makes the box around the problem answer and puts the content inside
}

\newcommand{\homeworkSectionName}{}
\newenvironment{homeworkSection}[1]{ % New environment for sections within homework problems, takes 1 argument - the name of the section
\renewcommand{\homeworkSectionName}{#1} % Assign \homeworkSectionName to the name of the section from the environment argument
\subsection{\homeworkSectionName} % Make a subsection with the custom name of the subsection
\enterProblemHeader{\homeworkProblemName\ [\homeworkSectionName]} % Header and footer within the environment
}{
\enterProblemHeader{\homeworkProblemName} % Header and footer after the environment
}
   
%----------------------------------------------------------------------------------------
%	NAME AND CLASS SECTION
%----------------------------------------------------------------------------------------

\newcommand{\hmwkTitle}{\"{U}bungsblatt \#6} % Assignment title
\newcommand{\hmwkDueDate}{Donnerstag,\ Dezember\ 9,\ 2017} % Due date
\newcommand{\hmwkClass}{Computergrafik\ WS 2017/2018} % Course/class
\newcommand{\hmwkClassTime}{} % Class/lecture time
\newcommand{\hmwkClassInstructor}{Gruppenabgabe} % Teacher/lecturer
\newcommand{\hmwkAuthorName}{Lukas Baur, \linebreak Felix B\"{u}hler, \linebreak Marco Hildenbrand} % Your name

%----------------------------------------------------------------------------------------
%	TITLE PAGE
%----------------------------------------------------------------------------------------

\title{
\vspace{2in}
\textmd{\textbf{\hmwkClass:\ \hmwkTitle}}\\
\normalsize\vspace{0.1in}\small{Due\ on\ \hmwkDueDate}\\
\vspace{0.1in}\large{\textit{\hmwkClassInstructor\ \hmwkClassTime}}
\vspace{3in}
}

\author{\textbf{\hmwkAuthorName}}
\date{} % Insert date here if you want it to appear below your name

%----------------------------------------------------------------------------------------

\begin{document}

\maketitle

%----------------------------------------------------------------------------------------
%	TABLE OF CONTENTS
%----------------------------------------------------------------------------------------

%\setcounter{tocdepth}{1} % Uncomment this line if you don't want subsections listed in the ToC

%\newpage
%\tableofcontents
\newpage

%----------------------------------------------------------------------------------------
%	Aufgabe 1
%----------------------------------------------------------------------------------------
\begin{homeworkProblem}

\textbf{Aufgabe 1.1}

\begin{itemize}
	\item Strahlendichte
	\item Flussdichte
	\item Einfallswinkel
\end{itemize}

\textbf{Aufgabe 1.2}

Er beschreibt wie stark sich Lichter in dem gegeben Material spiegeln lassen./
Das Verhältnis von einfallendem ausgehenden zu einfallendem licht.

\textbf{Aufgabe 1.3}

\begin{itemize}
	\item Gonioreflektometer:
	\subitem Photometer bewegt sich in gleichem Abstand um das Objekt und beleuchtet dabei das Objekt.
	\subitem Das ganze wird von einer Kamera aufgezeichnet 
	\item Phänomenologisch-motiverte Modelle:
	\subitem ist ein vollständig empirisches Modell und spiegelt nicht die Realität exakt wieder.
	\subitem Da es dem Energieerhaltungsatz widerspricht.
\end{itemize}

\end{homeworkProblem}

%----------------------------------------------------------------------------------------
%	Aufgabe 2
%----------------------------------------------------------------------------------------
\begin{homeworkProblem}
\textbf{Aufgabe 2.1}

\begin{itemize}
	\item Diffus:
	Streut den eintreffend Lichtstrahl in sämtliche ausgehende Richtungen.
	
	\item Glossy:
	Streut einen Teil der eingehenden Lichstrahlen in einem gewissen Winkel in Richtung des erwarteten Austritswinkels.
	
	\item Specular:
	ist sehr ähnlich zu spiegelnd, allerdings wirft es nur einen kleines Teil des eintreffenden Lichts zurück.
	
\end{itemize}

\textbf{Aufgabe 2.2}

In der Realität kommt durch die Reflektionen in einem beleuchteten Raum praktisch ueberall ein bisschen Licht an. Da die Beleuchtung in der CG aber vereinfacht werden soll, geht man von einer Grundhelligkeit aus. Diese Grundhelligkeit wird von der ambienten Komponente beschrieben.

Die Ambiente Komponente hat allerdings keine Relationen zu den "realen" physikalischen Vorgängen.

\end{homeworkProblem}

\end{document}
