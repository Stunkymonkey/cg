\documentclass[12pt,pdftex,a4paper]{article}
\usepackage[ngerman]{babel}
\usepackage[utf8]{inputenc}
\usepackage{amsmath}
\usepackage{amssymb}
\usepackage{ulem}
\usepackage{bbm}
\usepackage{array}
\usepackage{marvosym}
\usepackage{color}
\usepackage{hhline}
\newcommand{\bbN}{\mathbbm{N}}
\newcommand{\bbR}{\mathbbm{R}}
\newcommand{\bbZ}{\mathbbm{Z}}
\newcommand{\bbI}{\mathbbm{I}}
\newcolumntype{L}[1]{>{\raggedright\let\newline\\\arraybackslash\hspace{0pt}}m{#1}}
\newcolumntype{C}[1]{>{\centering\let\newline\\\arraybackslash\hspace{0pt}}m{#1}}
\newcolumntype{R}[1]{>{\raggedleft\let\newline\\\arraybackslash\hspace{0pt}}m{#1}}
\usepackage[pdftex]{graphicx}
\usepackage{listings}
\lstset{language=Python,basicstyle=\footnotesize}
\usepackage{pdfpages}
\usepackage{booktabs}
\PassOptionsToPackage{hyphens}{url}
\usepackage{hyperref}

\begin{document}
\title{ Grundlagen der Computergrafik,\\ Blatt 11}
\author{Lukas Baur, 3131138\\
		Felix Bühler, 2973410\\
		Marco Hildenbrand, 3137242}
\maketitle
\section*{Aufgabe 1}
\subsection*{1) Anzahl Dreiecke}

Es müssen mindestens $2(n-1)^{2}$ Dreiecke gerendert werden.
\subsection*{2) Anzahl Indices}

Dazu müssen 3 mal so viele Indices abgespeichert werden.

\section*{Aufgabe 3}
\subsection*{3.1)}
Der Elefant wirf gemäß des Lichteinstrahls nicht einen Schatten auf das ganze Terrain, eine Überdeckung des gesamten Terrains wäre unnötig.\\
Weiter entfernte Punkte nehmen sowieso einen sehr kleinen Wert an. Diese können vernachlässigt werden.
\subsection*{3.2)}
Dazu gibt es weitere Möglichkeiten:\\
- Ambient Occlusion \\
- Shadow Volumes \\
- Implicit Sphere Shadow Maps \\

\end{document}


